\subsection{白晋斌【计科】(MComp in AI @ National University of Singapore)}\label{baijinbin}

% 本科专业缩写请从以下挑选:计科、计金、计拔、AI、软院、电子
% 如果不属于上述院系,请自行填写
% 如果是硕士及以上,则专业缩写中无需填写本科专业,按 program name @ university 填写专业

\begin{table}[htbp]
    \centering
    \begin{tabular}{L{2cm}L{5.3cm}|L{2cm}L{5cm}}
    \toprule
    \multicolumn{4}{c}{\bf 个人背景与基本情况} \\
    \midrule

    {\bf 本科专业}  & 计算机科学与技术系 & 
    {\bf 申请方向}  & AI                            \\
    {\bf 硕士专业}  & N/A                            &
    {\bf 最终去向}  & MComp in AI @ NUS \\
    {\bf  GPA\&排名}& 4.49/5.00, 10\%               &  % 申请时提交的 GPA 与排名
    {\bf 性别}      & 男                            \\ % 男或女或N/A
    {\bf GRE}       & N/A     &
    {\bf 联系方式}  & WeChat: Mengzelev / QQ: 810594956(回复更快) \\
    {\bf TOEFL}     & 107 (R29 + L29 + S22 + W27)    &
    {\bf 邮箱}      & \href{mailto:jinbin5bai@gmail.com}{jinbin5bai@gmail.com} \\
    {\bf IELTS}     & N/A                            & 
    {\bf 个人网站}  & N/A                            \\
    {\bf Publications} & \multicolumn{3}{l}{N/A}     \\
    {\bf 推荐信组成}   & \multicolumn{3}{l}{许畅老师(课程推);蒋炎岩老师(课程+科研推);实习mentor(工业推)}     \\
    
    \midrule
    
    {\bf 申请情况:} \\
    {\bf\textcolor{mygreen}{offer}}     & \multicolumn{3}{L{13.2cm}}{
        MComp in AI @ NUS
    }\\
    {\bf\textcolor{myred}{reject}}    & \multicolumn{3}{L{13.2cm}}{
        SOC PhD @ NUS; ECE PhD @ NUS
    }\\
    
    \bottomrule
    \end{tabular}
\end{table}

\paragraph{申请时间线}

大一大二:由于被灌输过“出国申请GPA是最重要的”的观点,这段时间在一心上课。

大三:考英语,暑期实习。

大四上:网申。

大四下:选offer,办签证。

网申安排相关推荐参看{\bf 18年飞跃手册袁帅学长}。

具体timeline:
\begin{longtable}{l|l}
\hline
时间              & 事件                                          \\\hline
2019.7-2019.8   & 新东方托福培训班                                  \\
2019.11.2       & 一刷托福(南大鼓楼),107分手                                  \\
2020.6          & 微软暑期实习面试                                  \\
2020.7-2020.10  & 微软SDE实习                                     \\
2020.9.5          & 一战GRE(西交利物浦考场),320                                   \\
2020.10.11         & 二战GRE(南师大随园考场),327分手                                 \\
2020.11         & 开始浏览各校网站,准备网申                               \\
2020.11.10左右    & 开始准备网申材料(成绩单等)                              \\
2020.11.15左右    & 开始第一批网申(ddl 12.1)                           \\
2020.11.16      & 完成SoP草稿,交给芝士圈顾问修改                           \\
2020.11.19      & 拿到顾问一改文书                                    \\
2020.11.21      & 提交根据一改建议修改结果                                \\
2020.11.22      & 顾问提交最终修改结果                                  \\
2020.11.25      & 给老师发送第一批网申推荐信链接                             \\
2020.12.1       & 提交第一批网申                                     \\
2021.12.5左右     & 开始准备第二批网申(ddl12.15)                         \\
2021.12.5-12.15 & 填写第二批网申表格,这批学校真的巨多                          \\
2021.12.7 & 给老师发送第二、三批学校的推荐信链接\\ 
2021.12.15      & 提交第二批网申                                     \\
2021.1.10     & 提交第三批网申                                     \\
2021.1.14       & UIUC POI 面试                                 \\
2021.1.13-17    & 制作面试用slides                                 \\
2021.1.17       & UToronto POI 面试                             \\
2021.1.22       & UWarterloo POI 面试,被要求完成一篇paper critique     \\
2021.2.5        & UWaterloo 另一位POI 面试                         \\
2021.2.9        & 提交UWaterloo POI 要求的paper critique,拿到口头offer \\
2021.2.10       & McGill POI 面试                               \\
2021.2.11       & Wisc-Madison Rej                          \\
2021.2.12       & UBC ECE MEng AD + UCSD MSCS AD            \\
2021.2.22       & UToronto MScAC面试                            \\
2021.2.23       & UToronto ECE MSAC offer                   \\
2021.2.28       & 和Utoronto导师进行二次meeting,询问细节                 \\
2021.3.9        & 决定接受UToronto ECE MASC offer                 \\
2021.3.10       & UWaterloo MMath offer                     \\
2021.3.16       & UIUC MSCS Rej                             \\
2021.3.17       & UToronto CS MSc Rej                       \\
2021.3.10-4.2   & 准备签证材料                                      \\
2021.4.2        & UAlberta CS MSc Rej                       \\
2021.4.2        & 提交加拿大留学签证申请                                 \\
2021.4.14       & 签证申请通过,收到贴签信\\\hline                  
\end{longtable}

特别吐槽一下加拿大的网申系统。美国很多学校都是外包的同一个系统,以那个为参考来看,加拿大网申系统极其落后:信息传递慢、可能无法查询学校是否收到TG成绩、可能无法查询老师推荐信的填写状态、先交申请费后填推荐信导致推荐人信息无法修改、上传过的材料无法修改……总之需要特别小心,有不懂的地方就问小秘,猜错了后果可能很严重。

\paragraph{T, G 准备}
我TG都算比较顺利的,适合英语基础比较好的同学借鉴。一言以蔽之:能缩短的战线绝对不要拉长。快速解决掉英语能让你的申请道路明朗很多,专心应付其他环节。

加拿大大部分项目都不要求GRE,今年因为疫情美国很多学校也waive了GRE,但关于一个好的GRE成绩能起到多大作用目前还没有定论,我建议时间充足的话还是考一下。

个人经历(流水账):我的托福是在新东方报了班,结课之后自己坚持刷了2个月的题(每天一小时左右),第一次考抱着试一试的心态没想到考了107已经够用了,所以没受到疫情影响,当然如果没有疫情我可能还会想再考个高一点的分数。但GRE就不一样了,最开始约了20年2月的考试,后来一直循环被取消-再约,直到9月才考到。第一次考GRE的时候我单词基本没记住几个,做得非常崩溃,成绩也不理想只能二刷。实习期间上班工作下班背单词刷题,通勤地铁上还在背单词都快背吐了,度过了一段黑暗岁月,第二次考终于拿到了理想的成绩。

TG有个需要注意的地方就是时限,托福{\bf{2年}}、GRE {\bf{5年}},一般学校都会要求申请时甚至入学时成绩有效。GRE最短考试间隔为21天,托福最短考试间隔3天。因此推荐先考GRE,不要听信某些培训机构的谗言一早就考掉托福(除非有交换需求),我身边不少人都有过托福过期重考的教训。还有有些学校的TA会对托福{\bf{口语}}有要求,建议预先了解一下。

备考方法方面,我体感TG题目都比较套路化,有一定的英语底力+多刷题、多总结做题经验,就能获得一个够用的成绩。托福的听力和阅读就纯靠刷题,听力需要培养对关键词的敏感以及根据少量笔记/印象推测答案的能力,阅读就当长一点的高考阅读做;口语写作多练,多积累点句型和表达到临场能随便用的程度。GRE比托福更套路化。V需要的词汇量非常大,我个人的经验最有效的是{\bf{短时间内大量重复}}。虽然我单词从1月就开始背了,然而几乎所有的单词我都是在考前两周内记住的。单词书我推荐新东方的要你命小紫书(我没收广告费,酌情购买),比背词软件护眼,相比大绿书也少了很多记不住且没什么用的同近义词用法。Q可以靠天朝学生的底力硬扛过去,但我硬扛了两次都只有167,听说有学校不满170不收(未证实),所以有条件的话还是稍微做点题吧。

关于要考到多少:研究型项目够用就行,授课型项目越高越好。

关于要不要报班的问题,我觉得看个人。我报了新东方的托福班,觉得也就托福的口语和写作老师的点评服务比较有用。报班只是有概率能让你更高效地吸收别人总结的做题经验,但并该刷的题也还是得刷,也不能帮你提升英语底力。很多有用的备考资料网上都可以找到,也不一定要靠培训班。


\paragraph{NJU 实验室与导师}
大二开始在软件所SPAR组蒋炎岩老师手下做一个Empirical SE方面的课题,因为课业压力大和各种拖延进展很慢,现在变成了毕业论文课题,\sout{甚至觉得毕设也要做不完了}。详细内容见下一节科研经历。


\paragraph{科研经历}
太长不看版本:

\begin{itemize}
    \item 科研经历在申请中起到的作用:充实简历+要到科研推荐信+科研面试时有牛可吹+暑研说不定就转正了。
    \item 科研经历在申请外起到的作用:积累经验,提前体验,帮助你决定走科研or工作道路,帮助你选择研究方向
\end{itemize}

我科研经历算是比较少的,因为时间安排不当也没有找国外教授做暑研,比较后悔。大二开始跟蒋炎岩老师做了一个研究课题,断断续续有一点进展,勉强可以写进简历和做成slides在面试时吹牛,然后老师也答应帮我写推荐信(老师自己帮我写,而不是我写好了给他签字)。

科研经历对于申请研究型的项目是很重要的。老师决定要不要收你之前,基本都会和你约个Zoom meeting,让你聊一聊自己的各项经历。这时候科研经历就是你阐述的重点。关于本科期间如何选导师选项目,我建议考虑三个方面:{\bf{兴趣、老师给亲笔推荐信的意愿、项目中你个人能获得的贡献度}}。 

兴趣:计算机研究方向千千万万,也许确实存在你最感兴趣的那个方向,但能在本科有限的时间内找到的可能性微乎其微,因此我建议找到一个你感觉也许能够和平相处五六年的大方向就可以了。兴趣点可以从上过的课程中回忆寻找,优秀的课程都会涉及一些领域前沿的内容。我个人比较反对选方向只考虑前途而去选择“大热门”方向,没人知道你博士毕业之后会是什么行情。看到群里有小朋友“无脑随大流选AI”,显然没有经历过现实的毒打。实在担忧出路的话避开少数最坑的,剩下的其实都差不多。

老师给亲笔推荐信的意愿:最好的结果是能套到海外大牛的强推,国外在推荐信诚信这块很严格,因此国外推荐信含金量>>国内推荐信。但如果你跟我一样只有国内的科研经历,推荐信的重要性也不可小觑。导师/committee完全不了解你时通常会尽量利用有限的资料(CV、SoP、推荐信),所以即使是国内的推荐信,一封真实的强推也能给你增色不少。蒋老师告诉我我套的几个老师全都去问他了,可见老师愿意帮你说话也是很重要的,至少科研推是老师亲笔帮你写的话这时候就不会出现自相矛盾的尴尬情况了。

项目中能获得的贡献度:听说AI领域竞争很激烈,大家都为了paper卷破了头,SE/PL方向目前还没这么严重,本科期间发paper的很少。我不太喜欢这种风气,就提供点个人见闻仅供参考吧:我的研究生导师在一次聊天中告诉我,她面过不少有paper的学生,paper吹得天花乱坠,但关于在项目中做了什么工作、做这份工作有什么意义等问题,学生自己却说不清楚,相比之下她更喜欢自己而非老师主导的研究经历。

\paragraph{实习经历}

对于研究型项目来说,工业经历其实不太重要,写在简历上也就是作为你写代码能力的一个证明。如果目标已经明确是科研的话,极力推荐优先安排暑研。我是因为大三时候还没有完全决定将来工作还是科研才投了个实习感受了一下。

如果有计划找实习,建议大二暑假就开始刷leetcode。选择公司时建议优先外企,其次是国外有知名度的国内大厂。也许是由于疫情,现在公司普遍提供remote和onsite两种选择,我推荐onsite,虽然要解决租房、交通等一系列令人头大的问题,但可以实地感受企业文化,与同事和领导有更多的交流机会(搞好关系就是一封工业推),帮助你决定将来工作or科研。

大三疫情网课的那个学期,身边很多同学都去实习了,peer pressure使本来不打算实习的我硬着头皮在6月初投了苏州微软的SDE实习,然后幸运地过了,于是在苏州微软M365 Insights组写了三个月代码。公司环境相当宽松,午饭前和晚饭后都没什么人(我是不是暴露了什么);组里氛围很不错,mentor人非常nice;实习project写了一个网页应用也很有意思(组里不让实习生碰后端代码我只能写前端);感觉就像上了三个月web开发课,写大作业还能白嫖茶水间零食饮料+有工资拿。总之三个月过得相当愉快。

话虽如此,我最终还是选择了科研。在苏州微软的三个月确实很愉快,但我还是会在每天上同一个楼梯、坐同一个工位的时候感到一丝不安和厌倦,我觉得我可能无法接受三点一线几十年如一日的工作,比起钱更在意能不能做自己想做的东西。以及“挥霍”父母的钱财读授课项目实在于心不忍,所以还是去嗑盐了呀。


\paragraph{海外经历}
无。

我觉得对于申博选手来说,交换可以用来套对方学校的老师,否则去的意义不大。当然舍得花钱体验一下国外的学习生活也是可以的。

\paragraph{套磁经验}

流程:首先确定要套的老师。各大学校系主页上一般都会列出faculty和各自的研究方向,中筛选出我感兴趣的老师。也可以时刻关注各大留学论坛、公众号的招生广告版,既然发了广告那肯定是缺学生的老师。

浏览老师们的个人主页,记录下他们的研究方向和邮箱。然后按照感兴趣程度一一发邮件\sout{骚扰}。{\bf{注意有个别老师可能会要求你先提交学校申请再联系ta,也有老师会在个人主页上写套磁邮件需要包含某些要素以保证你阅读过ta的个人主页。}}如果老师对你也有兴趣,会回复邮件跟你约时间meeting(一般是Zoom)。这个就是传说中的科研面试,主要谈谈你本科期间各项经历,重点介绍科研经历,你要想方设法地跟老师“套近乎”,让ta觉得\sout{你会成为一个好的搬砖工}能和你愉快合作做出成果。以及建议往早里约时间,不要跟我一样拖到考试周之后。

套磁对象选择:我觉得需要考虑这两个因素:{\bf{具体研究方向、资历(Senior or AP)}}。

老师的具体研究方向可以通过个人主页来了解,阅读ta近几年的论文。一般个人主页上论文都会很多,可以对照着老师的projects看哪些是他参与度比较大的。来不及看完全篇可以只读abstract,其余部分快速翻过。对自己想申请的领域的顶会期刊也要比较熟悉,可以帮你一眼看出这个老师的研究方向和大致水平。这里的研究方向一定要具体,不能只是笼统的SE, PL, CV,AI之类的。我曾头铁套了个做theoretical PL的老师,他的论文读起来跟PL应用完全不是一个难度次元的,面试之前都快把我读哭了,引以为戒。

资历这个就是老生常谈了。往届飞跃手册基本都提到了。总结起来就是Senior不怎么管你,适合比较self-motivated很有想法的同学;AP比较hand on,现成的想法比较多,但也可能会比较push。不过这也只是一般规律,千万不要滥用成刻板印象,还是需要通过面试、询问学生等进一步了解的。

其实如果你不主动套磁,被你写在POI(Professor of interest)里的老师也有可能主动联系你,所以填POI的时候一定要谨慎。

措辞:这个1p3a等论坛上有很多参考,多看看就大概知道怎么写了。我放个我自己的写法仅供参考:
\begin{itemize}
    \item 开门见山:开头一两句话,介绍自己是谁,为什么要发这封邮件
    \item 套近乎:解释一下你是怎么知道老师的,为什么会对老师的研究感兴趣,一定要自然且真实。比如你表示看了老师的某篇论文之后很感兴趣,那这篇论文你必须是真的从头到尾看完过并且真的觉得有意思。能针对论文提出1-2个有价值的问题是最好的,但提不出不要硬挤。某位不愿意透露姓名的老师说,老板对学生是降维打击,是不是糊弄老师一眼就能看出来。(我现在回去看我的套磁邮件的这一部分真是尴尬极了......)
    \item 自我介绍详细版:稍微展开说一下你的背景,拿出最值得说的1-2个方面,比如高GPA、和老师研究方向相关度高的科研经历等。
    \item 研究计划:简要说一下你为什么想读研究生,研究生期间想做什么、为什么要找ta做导师。这一段也是一个总结段,意在表示“你想要的”和“老师能给你的”重合了,和写文书思路很像。
    \item 套话总结:感谢老师百忙之中抽出时间读你的邮件,并对和老师进一步联系表示一下期待
\end{itemize}

因为拖延症严重,我12月底、网申基本结束的时候才开始套磁。千万不要学我!记得早点开始!反过来说,如果你在觉得套磁已经来不及的时候看到这段文字,那就快点去套,只要还没收到拒信就有希望。

\paragraph{CV \& SoP}

CV是对你申请资本的一个简要概括,用语要精炼。科研项目的介绍措辞多问问带你做项目的导师和学长学姐,其他人都是门外汉。我不推荐找中介改内容,除非中介对你的那个具体领域特别了解。注意用词和语法正确。

SoP重要性详见1p3a帖子:\href{https://www.1point3acres.com/bbs/thread-581428-1-1.html}{作为志愿者审MS申请材料那些事}(申请前看到这篇文章就是赚到)。

SoP怎么写网上教程非常多搜索即得,我就不多嘴了,提一点情怀(?)上的东西。我认为SoP要做到{\bf{正确+诚恳}}。我没试过中介代写,不过我个人不推荐这么做。

正确性可以花点小钱(相比代写)找人润色。我找的是学长推荐的芝士圈的Henry. L,我觉得人很好,改得也很好,但价格有点贵。没收广告费,意见仅供参考。听说也有同学在闲鱼/国外同类网站上找的人,价格会便宜很多。

诚恳上需要下狠功夫。主旨是“你有什么”\&“学校要什么”、“你要什么”\&“学校有什么”的有机统一。要写出自然而通顺的SoP,需要不断扪心自问,把你整个读研源动力自始至终都摸清。为什么会对科研感兴趣,想要做什么样的科研。这些问题即使你现在糊弄过去了,将来也总有一天会逮住你。中介只会从你对一些问题的回答来揣测你,他们懂个锤子的你,只有你最懂你自己。我开始写SoP的时候其实还在犹豫我到底是工作还是科研,痛苦地沉思了一周,空间里还可以翻到我那时写的非主流小作文黑历史,纠结了一周终于想通理顺了,SoP写出来了,生活也明朗了许多。

注意,骑墙是大忌。如果你申科研项目,SoP就要表现你对科研的决心;工作同理。科研授课项目混申的话就需要准备两套文书。

提供一个小技巧,在文章结构上把需要改动的部分解耦,单独成段。例如我单独用一段来夸学校和项目,这样提交给不同的学校只需要改这一段。

觉得应该会有人需要所以我贴一下我文书的结构,括号内是大致词数:
\begin{itemize}
    \item 开门见山(40):我是谁,我是来申请xxx的
    \item 最早的源动力(100):我为什么喜欢cs。(When I was a child已经用烂了,但似乎大家都想不出别的说法。有更巧妙的想法可以加分。
    \item 研究经历(300)
    \item 实习经历(150)
    \item 现在的动力(200):我的经历如何使我决定走科研道路、选择做SE方向的科研
    \item 夸夸学校(130)
    \item 套话总结(40):你们学校真是太适合我辣,希望你们看看我。
\end{itemize}



\paragraph{选校标准}
我因为主申一共没几所名校的加拿大,所以选校方面没遇到什么困难。历年加拿大申请参考很少,这部分我主观因素相当大,酌情阅读。

先放一篇地里的文章在这里:\href{https://www.1point3acres.com/bbs/viewthread.php?tid=252&highlight=}{胡侃加拿大学校CS(计算机科学)系}。

研究型项目一般优先级老师>学校。加拿大可选范围其实很小,我觉得南大学子可能看得上的也就以下几所:

\begin{itemize}
    \item University of Toronto(U of T):加拿大top1,综合性大学,影响力不亚于美帝top20。学校在多伦多市中心。除了传统研究和授课硕士外还提供一个MScAC的授课项目,自带实习,转正率90\%以上,就业相当好。
    \item University of Waterloo:工科强校,cs本科相当强,研究生院貌似多大更胜一筹。学校在Waterloo比较偏。综排一般,国内知名度不高。毕业生在加拿大找工作很受欢迎。ECE MEng听说有点坑。
    \item Univeristy of British Columbia(UBC):也是强校。在温哥华,气候非常好,环境宜居。ECE MEng基本只要申就给你发AD。所在BC省是旅游大省,工业不发达,就业可能不如多大和Waterloo,但有移民优惠。
    \item McGill University:综排强校,CS专排一般,ML方向还不错。校内用英语但学校位于法语区。
    \item University of Alberta(UA):深度学习强校。靠近北极非常冷。位于石油大省卡尔加里,当地就业环境一般。
    \item SFU:近年来有上升趋势,和浙大有合作项目,听说授课项目就业不错。也在BC省。
    \item 剩下的York, Queens, Concordia, Dalhousie之类的我觉得你南学子应该都看不上了吧.....(我怕没学上差点就申了)
\end{itemize}

美国我也申了几所不过选得相当随意,就不解释了。

\paragraph{选\ offer 考量}

我主要考虑了以下几个方面,重要程度总体递减。

\begin{itemize}
    \item 老师的研究方向和research taste:个人主页、论文、面试交流
    \item 老师为人和mentoring style:面试交流、询问往届学生(即使是新AP一般博士期间也会带过学生)
    \item 老师的能力、经验、资历、人脉:履历、询问往届学生、询问NJU老师。
    \item 前途出路(工业界/学术界):面试交流、同项目alumni去向、论坛淘帖、询问NJU老师/业内人士
    \item 学校该方向实力:csrankings(计分标准不是非常合理,仅供参考)、顶会顶刊论文发表量
    \item 学校名声 (如果有回国打算):凭感觉
    \item 学校地域环境:论坛淘帖、询问在学校城市生活过的人。毕竟要待比较久,生活环境会很影响心情。
\end{itemize}

我最后主要是在Waterloo和Toronto中间纠结,咨询了身边很多人,包括NJU导师、同学、在加拿大的亲戚、老师带过的学生。虽然多大老师是新AP可能面临经验不足的问题,但多大老师的研究方向我更感兴趣一点,多大的名声在国内比Waterloo响,多大学校在市中心而Waterloo在乡下(我受够仙林了!),所以最后决定去了多大。

以及我稍微解释以下为什么我去了听上去有点掉价的ECE而不是CS。研究型项目导师是需要出钱资助学费+大部分生活费的,学费的收费标准各专业不同,且国际生会比当地学生贵(所以导师收国际生门槛一般都会更高)。我导师在CS和ECE都有招生名额,但是ECE的国际生收费比CS便宜,所以国际生她都录到了ECE。我也反复确认过多大ECE和CS的研究型项目在课程、研究上都是没有区别的,出路上比起你的专业出生也会更在意你做了什么,所以就放心去了ECE。注意这只是针对多大,有些学校ECE的课程会偏硬。录取难度上,研究型项目决策权在导师手中两专业区别不大,但授课项目一般学校ECE的门槛都会低于CS,怕申不到CS的话不妨试试ECE。

前面写太多了,这一部分中期应该没什么人关心吧,以后再详细写吧(咕咕咕


\paragraph{常见问题 \hyperref[optional]{\faLink}}

今年飞跃手册新增了这一部分,回答一些大家共同关注的问题。这些问题的详细说明收录在附录中,可以通过小标题中 \faLink 图标到达。\\
大家可以选择性的回答部分问题,也可以只回答一个问题的部分内容。回答问题格式如下样例所示。\\
尽可能不要修改附录中已有的问题。如有意愿添加问题,请提前联系Zangwei Zheng。\\

{\bf 【中介选择】} 
我个人认为只有两类人需要请全包中介:1. 大四上有更重要的事情需要忙,无暇顾及申请;2. 预算充足想买个全程顾问,俗话钱多。

从我的角度来看,花钱请中介并不能使我省心。如果我不自己足够了解申请这块,就会被中介利用信息不对称往死里忽悠。中介代写文书的弊端在文书一节已经提到。中介选校上,美国选校可能确实很困难我不宜多谈。中介网申,万一搞错了什么你可能除了暴打一顿并把ta挂上留学论坛意外别无他法......

早点(大二大三)请中介也许可以催你做一些事治拖延症,但我觉得还不如找个学长学姐,收费保证便宜好多(

{\bf 【国内外(及各国)比较】} 

作为还没开始国外读研、也不可能在国内读研的人,国内外比较的问题我没什么发言权。申请前我曾找一些老师聊过出国的问题,基本都表示即使不论排名,目前发达国家的高等教育普遍都比国内更先进完善,值得出去体验一下。我也觉得一直驻留在国内作为人生来说未免有点遗憾,所以决定利用留学的机会出去看看。

各国比较我主要比较下美国和加拿大吧。CS基本都会去美帝,美帝毕竟cs top1,技术岗收入也相当可观,也意味着竞争非常激烈,工签和移民都相当困难。加拿大大概算是一个不那么卷的地方,工资跟美帝没法比,高税收高福利,移民政策很宽松,生活压力相对小一些。而且近年来对美帝种族和阶级问题的负面报道很多,新冠更是让形势雪上加上,加拿大的文化就包容很多,也更加安全。

我选择加拿大还因为加拿大搞笑普遍提供研究型硕士项目。既和phd一样以做科研为主、院系导师全额资助,又只需要两年。如果和导师合作愉快,一年半的时候可以转phd,不然也可以直接毕业,有些学校甚至支持直接转成授课硕士毕业了去工作。很适合不太敢直博但又想做科研的同学。录取方式同phd,决定权主要在导师。美国据我所知只有少数学校有类似项目,如UIUC。

以及我和一个新加坡国立大学Phd的老师面试聊天的时候,老师表达了一下他的观点:目前SE领域国内的研究水平已经不输国外了;但如果想往教职方向发展的话,欧洲和北美学校比起大陆、香港、新加坡等地的phd比较容易找到教职。


% {\bf 【国外读研情况】}  

\paragraph{申请季经验总结}

发现自己一不留神写了好多字啊......

生怕随意总结过于片面,所以就把自己的故事写在这里了,各自选择自己喜欢的方式解读吧w

如果还有问题可以私聊问我,个人QQ用得更多。祝大家都能拿到梦校offer!

\newpage

